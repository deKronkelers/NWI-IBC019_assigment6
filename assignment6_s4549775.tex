\documentclass[12pt]{article}

\usepackage{amssymb}
\usepackage{multicol}
\usepackage{enumerate}

\setlength\parskip{1em}
\setlength\parindent{0em}

\title{Assignment 6}

\author{Hendrik Werner s4549775}

\begin{document}
\maketitle

\section*{8.1}
\begin{enumerate}[a]
	\item %a
	The virtual address is split into a page number, a frame number, control bits, and an offset.

	The page number is used by the MMU to look up the physical address of the page in the page table. The frame number is added as an offset ontop of this to get the physical address of the frame which is then used together with the offset to determine the physical location of the virtual address.

	\item %b
	Because it is not explicitly stated I assume the the first number is the virtual page number and the number after the comma is the offset. I used the page size to determine

	\begin{enumerate}[i]
		\item %i
		The physical address is $7 * 1024 + 52 = 7220$.

		\item %ii
		There is no physical address for this virtual address as the valid bit is set to $0$ and the page frame number is empty.

		\item %iii
		The physical address is $0 * 1024 + 499 = 499$.
	\end{enumerate}
\end{enumerate}

\section*{8.2}
\begin{multicols}{3}
	\begin{enumerate}
		\item A[0] - A[3]
		\item A[3] - A[6]
		\item A[7] - A[11]
		\item A[12] - A[15]
		\item B[0] - B[3]
		\item B[3] - B[6]
		\item B[7] - B[11]
		\item B[12] - B[15]
		\item C[0] - C[3]
		\item C[3] - C[6]
		\item C[7] - C[11]
		\item C[12] - C[15]
	\end{enumerate}
\end{multicols}

\begin{enumerate}[a]
	\item %a
	The process has a $4$ page working set and one page is occupied by the program code itself so we have $3$ pages left for the data. On each step we need all $3$ arrays so we can only load one page of each into the working set at any time.

	Because of the way the arrays are split up into pages (see schema above) you can access, for exmaple, A[0; 0] upto A[3, 63] without a page fault but the program given does not acess them in this order.

	It loops through the columns rather than the rows and so on each $4$th iteration we get $3$ page faults (one for each array). There is a total of $64^2$ iterations so there is a total of $\frac{64^2}{4} * 3 = 3072$ page faults.

	\item %b
	The program could be significantly improved by iterating over the rows rather then the columns. This can be done by swapping the $i$ and $j$ looping variables.

	\item %c
	The schema shows that we only get page faults every four rows when iterating over the rows. This means we only get $3$ page faults every $64 * 4 = 256$ iterations.

	Switching $i$ and $j$ reduces the number of page faults to $\frac{64^2}{256} * 3 = 48$.
\end{enumerate}

\section*{8.4}
I assume that we also have a $3$ page set as in figure 8.15 from the book so this is the initial state:

\begin{tabular}{|p{2em}|}
	\hline
	\\
	\hline
	\\
	\hline
	\\
	\hline
\end{tabular}

\begin{enumerate}[a]
	\item %a
	\begin{multicols}{3}
		\begin{description}
			\item[7]
			\begin{tabular}{|p{2em}|}
				\hline
				7 \\
				\hline
				\\
				\hline
				\\
				\hline
			\end{tabular}
			\item[0]
			\begin{tabular}{|p{2em}|}
				\hline
				7 \\
				\hline
				0 \\
				\hline
				\\
				\hline
			\end{tabular}
			\item[1]
			\begin{tabular}{|p{2em}|}
				\hline
				7 \\
				\hline
				0 \\
				\hline
				1 \\
				\hline
			\end{tabular}
			\item[2]
			\begin{tabular}{|p{2em}|}
				\hline
				2 \\
				\hline
				0 \\
				\hline
				1 \\
				\hline
			\end{tabular}
			F
			\item[0]
			\begin{tabular}{|p{2em}|}
				\hline
				2 \\
				\hline
				0 \\
				\hline
				1 \\
				\hline
			\end{tabular}
			\item[3]
			\begin{tabular}{|p{2em}|}
				\hline
				2 \\
				\hline
				3 \\
				\hline
				1 \\
				\hline
			\end{tabular}
			F
			\item[0]
			\begin{tabular}{|p{2em}|}
				\hline
				2 \\
				\hline
				3 \\
				\hline
				0 \\
				\hline
			\end{tabular}
			F
			\item[4]
			\begin{tabular}{|p{2em}|}
				\hline
				4 \\
				\hline
				3 \\
				\hline
				0 \\
				\hline
			\end{tabular}
			F
			\item[2]
			\begin{tabular}{|p{2em}|}
				\hline
				4 \\
				\hline
				2 \\
				\hline
				0 \\
				\hline
			\end{tabular}
			F
			\item[3]
			\begin{tabular}{|p{2em}|}
				\hline
				4 \\
				\hline
				2 \\
				\hline
				3 \\
				\hline
			\end{tabular}
			F
			\item[0]
			\begin{tabular}{|p{2em}|}
				\hline
				0 \\
				\hline
				2 \\
				\hline
				3 \\
				\hline
			\end{tabular}
			F
			\item[3]
			\begin{tabular}{|p{2em}|}
				\hline
				0 \\
				\hline
				2 \\
				\hline
				3 \\
				\hline
			\end{tabular}
			\item[2]
			\begin{tabular}{|p{2em}|}
				\hline
				0 \\
				\hline
				2 \\
				\hline
				3 \\
				\hline
			\end{tabular}
		\end{description}
	\end{multicols}
	\item %b
	\item %c
	\item %d
	\item %e
\end{enumerate}

\end{document}
